%The word �Abstract� should be centered 2? below the top of the page. 
%Skip one line, then center your name followed by the title of the 
%thesis/dissertation. Use as many lines as necessary. Centered below the 
%title include the phrase, in parentheses, �(Under the direction of  
%_________)� and include the name(s) of the dissertation advisor(s).
%Skip one line and begin the content of the abstract. It should be 
%double-spaced and conform to margin guidelines. An abstract should not 
%exceed 150 words for a thesis and 350 words for a dissertation. The 
%latter is a requirement of both the Graduate School and UMI's 
%Dissertation Abstracts International.
%Because your dissertation abstract will be published, please prepare and 
%proofread it carefully. Print all symbols and foreign words clearly and 
%accurately to avoid errors or delays. Make sure that the title given at 
%the top of the abstract has the same wording as the title shown on your 
%title page. Avoid mathematical formulas, diagrams, and other 
%illustrative materials, and only offer the briefest possible description 
%of your thesis/dissertation and a concise summary of its conclusions. Do 
%not include lengthy explanations and opinions.
%The abstract should bear the lower case Roman number ii (if you did not 
%include a copyright page) or iii (if you include a copyright page).

\begin{center}
\vspace*{52pt}
{\Large \textbf{ABSTRACT}}
\vspace{11pt}

\begin{singlespace}
Natalie Stanley: Adapting community detection approaches to large, multilayer, and attributed networks  \\
(Under the direction of Peter J. Mucha)
\end{singlespace}
\end{center}

Networks have become a common data mining tool to encode relational definitions between a set of entities. Whether studying biological correlations, or communication between individuals in a social network, network analysis tools enable interpretation, prediction, and visualization of patterns in the data. Community detection is a well-developed subfield of network analysis, where the objective is to cluster nodes into `communities' based on their connectivity patterns. There are many useful and robust approaches for identifying communities in a single, moderately-sized network, but the ability to work with more complicated types of networks containing extra or a large amount of information poses challenges. In this thesis, we address three types of challenging network data and how to adapt standard community detection approaches to handle these situations. In particular, we focus on networks that are large, attributed, and multilayer. First, we present a method for identifying communities in multilayer networks, where there exist multiple relational definitions between a set of nodes. Next, we provide a pre-processing technique for reducing the size of large networks, where standard community detection approaches might have inconsistent results or be prohibitively slow. We then introduce an extension to a probabilistic model for community structure to take into account node attribute information and develop a test to quantify the extent to which connectivity and attribute information align. Finally, we demonstrate example applications of these methods in biological and social networks. This work helps to advance the understand of network clustering, network compression, and the joint modeling of node attributes and network connectivity. 
\clearpage
